


\documentclass[9pt]{article}


\usepackage[utf8]{inputenc}
\usepackage[T1]{fontenc}
\usepackage[polish]{babel}
\usepackage{array}
\usepackage{booktabs}
\usepackage{geometry}

\usepackage{enumitem}
\geometry{margin=0.5in}


\begin{document}
	
	\subsection*{ Argumenty o silnym wpływie lobby bankowego na politykę kryptowalutową w Polsce}
		\vspace{2em}
	
	\renewcommand{\arraystretch}{1.3}
	\begin{tabular}{@{} p{0.3cm} p{6cm} p{12cm} @{}}
		\toprule
		\textbf{Nr} & \textbf{Skrót argumentu} & \textbf{Dlaczego wskazuje na wpływ banków?} \\
		\midrule
		1 & Celowe opóźnianie wdrożenia MiCA & Polska spóźnia się z wprowadzeniem unijnych przepisów MiCA (miała to zrobić do 30 grudnia 2024). Podczas gdy Niemcy i Estonia już wydały pierwsze licencje dla firm krypto, Polska wciąż nie uchwaliła odpowiedniej ustawy. To daje bankom czas, by utrzymać przewagę i ograniczyć konkurencję ze strony firm krypto i fintechów. \\
		
		2 & Projekt ustawy “anty-krypto” & Rządowy projekt z marca 2024 przewiduje, że KNF może zablokować konto inwestora krypto na 96 godzin, a potem przedłużyć blokadę aż do 6 miesięcy – bez decyzji sądu i możliwości odwołania. Takie rozwiązanie chroni interesy banków, które boją się, że pieniądze ludzi trafią na giełdy krypto, a nie do nich. \\
		
		3 & Masowe blokady kont i przelewów związanych z krypto & Banki (np. Santander, mBank, PKO BP) wrzucają firmy krypto na listę “niechcianych branż”, zamykają konta lub wstrzymują przelewy, powołując się na przepisy o praniu pieniędzy (AML). Państwo nic z tym nie robi – co sprzyja bankom, bo krypto przestaje być realną konkurencją. \\
		
		4 & Wspólne kampanie zniechęcające do krypto & KNF i NBP od dawna prowadzą kampanie ostrzegające przed kryptowalutami, a w listopadzie 2024 KNF znowu ogłosiła, że krypto to “ryzyko”. Związek Banków Polskich (ZBP) szkoli banki z MiCA i razem z KNF i NBP tworzy jednolite, nieprzychylne stanowisko w rozmowach z rządem. \\
		
		5 & Przechodzenie polityków do banków & Były minister finansów Mateusz Szczurek w 2024 r. trafił do zarządu banku BGK – za zgodą KNF. To pokazuje, że politycy mogą łatwo przechodzić z rządu do banków, co tworzy bliskie powiązania i ułatwia lobbing. \\
		
		6 & Rekordowe zyski banków – spore pieniądze na lobbying & W 2024 roku banki w Polsce zarobiły rekordowe 42,2 mld zł. Takie zyski pozwalają im finansować kampanie medialne, ekspertyzy i raporty, które wspierają ich interesy, a jednocześnie mogą być niekorzystne dla rynku krypto. \\
		\bottomrule
	\end{tabular}
		\vspace{2em}
\subsection*{Jak te elementy składają się w spójną narrację?}
	\vspace{2em}
\begin{itemize}[left=0pt, itemsep=1em]
	
	\item \textbf{Ustawodawczy paraliż} – mimo nacisku Komisji Europejskiej Polska jest w ogonie UE z implementacją MiCA. Każdy miesiąc zwłoki utrzymuje przewagę tradycyjnych banków nad giełdami i fintechami.
	
	\item \textbf{Regulacje defensywne zamiast proinnowacyjnych} – zamiast prostych licencji krypto, rząd proponuje przepisy przewidujące blokady aktywów, co w praktyce odstrasza firmy od lokowania się w Polsce.
	
	\item \textbf{De-banking krypto-inwestorów} – gdy bank może bezkarnie zablokować twoją wypłatę z Binance, a ustawodawca nie reaguje, to sygnał, że interes sektora bankowego jest ważniejszy niż rozwój innowacji.
	
	\item \textbf{Jednolity przekaz ryzyka} – banki, NBP i KNF mówią jednym głosem o “zagrożeniach”, wzmacniając społeczne i polityczne przyzwolenie na twarde regulacje.
	
	\item \textbf{Kongruencja personalna} – gdy ci sami ludzie rotują między ministerstwem finansów, komisją nadzoru i zarządami banków, naturalnie przenoszą swoje preferencje regulacyjne.
	
	\item \textbf{Finansowa siła nacisku} – rekordowe zyski dają środki na think-tanki, raporty, kampanie medialne i konsultacje publiczne – miękkie formy lobbingu, które rzadko są transparentne.
	
\end{itemize}


	\noindent \textbf{Wniosek:} Zbieżność tych faktów tworzy mocny łańcuch poszlak wskazujących, że lobby bankowe skutecznie torpeduje pro-kryptowalutowe inicjatywy ustawodawcze w Polsce, zachowując uprzywilejowaną pozycję sektora tradycyjnych finansów.

	
\end{document}
